\chapter{Conclusões}

Nesta dissertação foi desenvolvido um comutador de Ethernet com suporte para a atuação como o elemento de rede denominado \textit{Transparent Clock} definido na norma do protocolo PTP. Numa primeira fase, estudou-se várias formas de implementar os elementos que constituem o comutador com foco em encontrar uma solução que permitisse optimizar o débito do mesmo com os recursos e tempo disponibilizados para o seu desenvolvimento. Seguidamente, foi também estudada robustamente a versão mais recente da norma que define o PTP. Após uma análise exaustiva foi decidido que o ponto de partida adequado seria a implementação do modo \textit{One-Way} dos mecanismos P2P e E2E. Após a decisão da arquitectura e das funcionalidades a implementar procedeu-se à elaboração de uma descrição em Verilog do circuito em formato RTL que implemente o comutador. Devido à complexidade da totalidade do modelo a descrever, esta tarefa requereu uma quantidade de tempo considerável daquele usado nesta dissertação. 

\section{Trabalho futuro}

Ainda que tenham sido finalizado com sucesso os objetivos pretendido com esta dissertação, o dispositivo desenvolvido continua longe de estar completo, e a aprovação em 2023 do PRR associado garante a continuidade do desenvolvimento aqui iniciado. A trajetória a tomar após a conclusão do trabalho realizado vai estar dependente das decisões tomadas pelos futuros responsáveis do mesmo, no entanto, considerando a indústria na qual se incluirá este dispositivo, deverá seguir uma das seguidamente mencionadas. \par

No que se refere ao PTP há muitas funcionalidades que não foram ainda implementadas. Além do funcionamento como \textit{Transparente Clock}, poderá ser do interesse dos detentores do comutador que este atue também como \textit{Ordinary Clock}. Para tal, será necessário executar o BMC e proceder ao cálculo dos tempos de propagação entre \textit{Master} e \textit{Slave} com recurso ao envio e manipulação das mensagens \textit{Sync}, \textit{Follow\_Up}, \textit{Delay\_Req} e \textit{Delay\_Resp}. A opção tomada de recorrer às estruturas de dados e módulos sugeridos na norma para implementação de uma instância prevê uma maior agilidade numa inclusão futura do suporte para funcionamento como \textit{Ordinary Clock}. Menos relevante, mas também possível de inclusão, é o suporte para o processamento das mensagens \textit{Signaling} e \textit{Management}, e das várias \textit{TLV}'s definidas na norma.
Os sistemas de automação de subestações são elementos fulcrais da indústria energética, por isso os seus serviços devem ser disponibilizados ininterruptamente. Tal pressupõe que as comunicações entre os dispositivos constituintes da mesma não devem ser interrompidas devido a falhas nos elementos responsáveis pelas comunicações. Para o evitar a rede de comunicações sobre Ethernet deve implementar protocolos preparados para providenciar caminhos de comunicação alternativos em caso de falha de um elemento de rede. Os dois mais conhecidos são o HSR e o PRP, e a implementação no comutador do suporte para um dos dois algoritmos será quase inevitável. \par 
Por último, é necessário cumprir com os requisitos de segurança comuns na indústria. Assim, deverá ser também implementado o MACsec(\textit{Media Access Control Security}) \cite{MACSec}. Este é um protocolo de segurança operado em redes LAN baseado no algoritmo Galois de encriptação recorrendo a chaves simétricas. O MACsec permite garantir a integridade e confidencialidade das tramas circulantes na rede. O MACsec não requer a adição de uma quantidade excessiva de recursos quando implementado em \textit{hardware}, e por essa razão a sua implementação no comutador será um passo natural.