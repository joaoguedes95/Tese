%%%%%%%%%%%%%%%%%%%%%%%%%%%%%%%%%%%%%%%%%%%%%%%%%%%%%%%%%%%%%%%%%%%%%%%%
%                                                                      %
%     File: Thesis_Resumo.tex                                          %
%     Tex Master: Thesis.tex                                           %
%                                                                      %
%     Author: Andre C. Marta                                           %
%     Last modified :  2 Jul 2015                                      %
%                                                                      %
%%%%%%%%%%%%%%%%%%%%%%%%%%%%%%%%%%%%%%%%%%%%%%%%%%%%%%%%%%%%%%%%%%%%%%%%

\section*{Resumo}

% Add entry in the table of contents as section
\addcontentsline{toc}{section}{Resumo}

O PTP é um protocolo apresentado pela IEEE para auxiliar na sincronização dos relógios internos de diferentes dispositivos a comunicar numa rede. Para isso, os dispositivos a sincronizar trocam entre si mensagens portadoras da informação do instante de transmissão entre as mesmas. Efectuando cálculos com os referidos instantes de transmissão, os dispositivos conseguem descobrir o tempo de propagação das mensagens entre os mesmos e com isso ajustar adequadamente os seus relógios internos \par
Uma segunda iteração da norma foi introduzida em 2008 denomindada PTPV2. A sua principal vantagem foi a introdução do \textit{Transparent Clock} que é um modo de execução do PTP pensado para dispositivos de reencaminhamento nas redes como comutadores e roteadores. A implementação nestes do modo de funcionamento como \textit{Transparent Clock} permite aos restantes dispositivos compensarem a variabilidade dos atrasos das mensagens nos primeiros aumentando a precisão da sincronização. \par
Esta tese apresenta uma arquitectura para um comutador de Ethernet com suporte para o funcionamento como \textit{Transparent Clock}, visando a sua instalação numa subestação elétrica. O comutador deverá mais tarde ser complementado com outros protocolos relevantes na indústria de distribuição de energia gerando um produto com potencial comercial que se encontra em falta no mercado. 




\vfill

\textbf{\Large Palavras-chave:} Ethernet, comutador, Verilog, PTP, FPGA

